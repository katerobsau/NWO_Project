\documentclass[10pt,a4paper]{article}

\begin{document}


Make navigation links a separate document so that I don't need to change it for each page.

\textbf{Problem Specification}
\begin{itemize}
	\item[DONE] remove risk region figure ? or add in mathematics 
	\item fix figure referencing
	\item [DONE] typo evenet
	\item Dirk said the King tide example wasn't realistic (ask Pier) or redo in terms of normal tide
	\item [DONE] this is not a word and is spelt wrong extemeness
	\item [DONE] fix the last paragraph in rethinking the problem bit
	\item[DONE] move the event definition stuff to another section
\end{itemize}

\textbf{Sea Level Metadata and Preprocessing}
\begin{itemize}
	\item Make hyperlinks names consistent 
	\item And Summary information about the data
	\item Move Harlingen Observations and graph to this section
	\item Update the graph, build the site again 
	\item Also need to clean the functions in the tidalHelpers directory, ie add in clusters
	\item Document the functions (starting to have enough I need to)
\end{itemize}

\textbf{Rainfall data}
\begin{itemize}
	\item Page is very much under construction (maybe comment it all out til ready to deal with it)
\end{itemize}

\textbf{Surge Climatology}
\begin{itemize}
	\item Relabel as sluice operation
	\item Add in details of minimum period between and above used to define clusters
	\item Shift discussion of missing data below first set of plots and expand (I need to revist sources of missingness)
	\item Remove twelve hours and just say periodicity
	\item Would like to change the shading, to something like the largest period of risk during the surge event - worried this colouring is deceiving
\end{itemize}

\textbf{Surge Seasonality}
\begin{itemize}
	\item Relabel as climatology (?)
	\item Relabel axis so it is clear (? height)
	\item Also want plots of average risk period  duration
\end{itemize}

\textbf{Baseline: TODAY} 
\begin{itemize}
	\item[DONE] Spaghetti plots (tide and sur)
	\item[DONE] move the mathematical forecast description  definition 
	\item where t is 
	\item m in 1.. 50
	\item Describe NGR (univariate method baseline)
	\item [DONE] Rank Histograms (plot and details)
	\item CRPS (make symmetry point re mean and median) (plot and details)
	\item Send to Kiri to double check
	\item explain we will do the multivariate soon 
\end{itemize}

\textbf{Event based Verification: TOMORROW AFTERNOON}
\begin{itemize}
	\item [DONE] A how high is the peak
	\item Add in the results from post-processing vs raw for these
	\item Evaluate with a bootstrapped BSS
\end{itemize}

\end{document}